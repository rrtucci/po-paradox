\documentclass[12pt]{article}
\usepackage{graphicx}     %standard package. Note graphics<graphicx<epsfig
\usepackage{color}     %standard package
\usepackage{hyperref} %standard package written by Sebastian Rahtz
%\usepackage{subeqn}    %allows equations 1a, 1b
%\usepackage{subfig}    %allows figures 1a, 1b
\usepackage{amsmath}
\usepackage{amssymb}
\usepackage{xurl}
\usepackage{bbm}%for indicator function
%\bibliographystyle{alpha}


\usepackage[nottoc]{tocbibind}

\usepackage[color,matrix,frame,arrow,curve]{xy}
\usepackage[table,xcdraw]{xcolor}
\usepackage[shortlabels]{enumitem}
\usepackage{array}
\usepackage{pdflscape}
\usepackage{multicol, multirow}
\usepackage{ctable}
\usepackage{dirtree}
\usepackage{subfig}

%\usepackage{fdsymbol}
\usepackage{animate}% play animation

\usepackage[vlined,ruled]{algorithm2e}
\newcommand{\mycommfont}[1]{\small\ttfamily\textcolor{blue}{#1}}
\SetCommentSty{mycommfont}

%highlighting
\usepackage{mdframed}

\usepackage{pdfpages}
\usepackage{hyperxmp}

\paperheight=11in
\topmargin=0in
\headheight=0in
\headsep=0in
\topskip=0in
\textheight=8.5in
\footskip=.5in

\paperwidth=8.5in
\oddsidemargin=.25in
\evensidemargin=.25in
\textwidth=6.0in
\parindent=.5in

\newtheorem{claim}{Claim}
\newcommand{\proof}[0]{{\bf proof:} }
\newcommand{\qed}[0]{\newline \noindent{\bf QED }}


\newcommand{\bra}[1]{\langle#1|}
\newcommand{\ket}[1]{|#1\rangle}
\newcommand{\av}[1]{\left\langle#1\right\rangle}
\newcommand{\pder}[2]{\frac{\partial#1}{\partial#2}}
\newcommand{\tr}[0]{{\rm tr }}
\newcommand{\beq}{\begin{equation}}
\newcommand{\eeq}{\end{equation}}
\newcommand{\bsub}{\begin{subequations}}
	\newcommand{\esub}{\end{subequations}}
\newcommand{\beqa}{\begin{eqnarray}}
\newcommand{\eeqa}{\end{eqnarray}}
\newcommand{\rarrow}[0]{\rightarrow}
\newcommand{\larrow}[0]{\leftarrow}
\newcommand{\uarrow}[0]{\uparrow}
\newcommand{\darrow}[0]{\downarrow}
\newcommand{\Rarrow}[0]{\Rightarrow}
\newcommand{\nRarrow}[0]{\nRightarrow}
\newcommand{\Larrow}[0]{\Leftarrow}
\newcommand{\nLarrow}[0]{\nLeftarrow}
\newcommand{\ul}[1]{\underline{#1}}
\newcommand{\ol}[1]{\overline{#1}}
\newcommand{\ZZ}[0]{ {\mathbb{Z}}}
\newcommand{\RR}[0]{{ \mathbb{R}} }
\newcommand{\CC}[0]{{ \mathbb{C}} }
\newcommand{\ground}{{}_{\stackrel{\stackrel{\displaystyle{\bot}}{-}}{.}}}
\newcommand{\norm}[1]{\parallel#1\parallel}
\newcommand{\eqdef}[0]{\;\;\stackrel{\text{def}}{=}\;\;} 

\newcommand{\rva}[0]{{\ul{a}}}
\newcommand{\rvb}[0]{{\ul{b}}}
\newcommand{\rvc}[0]{{\ul{c}}}
\newcommand{\rvd}[0]{{\ul{d}}}
\newcommand{\rve}[0]{{\ul{e}}}
\newcommand{\rvf}[0]{{\ul{f}}}
\newcommand{\rvg}[0]{{\ul{g}}}
\newcommand{\rvh}[0]{{\ul{h}}}
\newcommand{\rvi}[0]{{\ul{i}}}
\newcommand{\rvj}[0]{{\ul{j}}}
\newcommand{\rvk}[0]{{\ul{k}}}
\newcommand{\rvl}[0]{{\ul{l}}}
\newcommand{\rvll}[0]{{\ul{\ell}}}
\newcommand{\rvm}[0]{{\ul{m}}}
\newcommand{\rvn}[0]{{\ul{n}}}
\newcommand{\rvo}[0]{{\ul{o}}}
\newcommand{\rvp}[0]{{\ul{p}}}
\newcommand{\rvq}[0]{{\ul{q}}}
\newcommand{\rvr}[0]{{\ul{r}}}
\newcommand{\rvs}[0]{{\ul{s}}}
\newcommand{\rvt}[0]{{\ul{t}}}
\newcommand{\rvu}[0]{{\ul{u}}}
\newcommand{\rvv}[0]{{\ul{v}}}
\newcommand{\rvw}[0]{{\ul{w}}}
\newcommand{\rvx}[0]{{\ul{x}}}
\newcommand{\rvy}[0]{{\ul{y}}}
\newcommand{\rvz}[0]{{\ul{z}}}


\newcommand{\rvA}[0]{{\ul{A}}}
\newcommand{\rvB}[0]{{\ul{B}}}
\newcommand{\rvC}[0]{{\ul{C}}}
\newcommand{\rvD}[0]{{\ul{D}}}
\newcommand{\rvE}[0]{{\ul{E}}}
\newcommand{\rvF}[0]{{\ul{F}}}
\newcommand{\rvH}[0]{{\ul{H}}}
\newcommand{\rvK}[0]{{\ul{K}}}
\newcommand{\rvL}[0]{{\ul{L}}}
\newcommand{\rvN}[0]{{\ul{N}}}
\newcommand{\rvQ}[0]{{\ul{Q}}}
\newcommand{\rvR}[0]{{\ul{R}}}
\newcommand{\rvT}[0]{{\ul{T}}}
\newcommand{\rvU}[0]{{\ul{U}}}
\newcommand{\rvV}[0]{{\ul{V}}}
\newcommand{\rvW}[0]{{\ul{W}}}
\newcommand{\rvX}[0]{{\ul{X}}}
\newcommand{\rvY}[0]{{\ul{Y}}}
\newcommand{\rvZ}[0]{{\ul{Z}}}


\newcommand{\rvtheta}[0]{{\ul{\theta}}}
\newcommand{\rvmu}[0]{{\ul{\mu}}}
\newcommand{\rvsig}[0]{{\ul{\sigma}}}
\newcommand{\rvDel}[0]{{\ul{\Delta}}}

\newcommand{\cala}[0]{{\cal A}}
\newcommand{\calb}[0]{{\cal B}}
\newcommand{\calc}[0]{{\cal C}}
\newcommand{\cald}[0]{{\cal D}}
\newcommand{\cale}[0]{{\cal E}}
\newcommand{\calf}[0]{{\cal F}}
\newcommand{\calg}[0]{{\cal G}}
\newcommand{\calh}[0]{{\cal H}}
\newcommand{\cali}[0]{{\cal I}}
\newcommand{\calk}[0]{{\cal K}}
\newcommand{\call}[0]{{\cal L}}
\newcommand{\calm}[0]{{\cal M}}
\newcommand{\caln}[0]{{\cal N}}
\newcommand{\calo}[0]{{\cal O}}
\newcommand{\calp}[0]{{\cal P}}
\newcommand{\calr}[0]{{\cal R}}
\newcommand{\cals}[0]{{\cal S}}
\newcommand{\calt}[0]{{\cal T}}
\newcommand{\calu}[0]{{\cal U}}
\newcommand{\calv}[0]{{\cal V}}
\newcommand{\calw}[0]{{\cal W}}
\newcommand{\calx}[0]{{\cal X}}
\newcommand{\caly}[0]{{\cal Y}}
\newcommand{\calz}[0]{{\cal Z}}

\newcommand{\lam}[0]{\lambda}
\newcommand{\Lam}[0]{\Lambda}
\newcommand{\alp}[0]{\alpha}
\newcommand{\eps}[0]{\epsilon}
\newcommand{\s}[0]{\sigma}
\newcommand{\su}[0]{{\Sigma}}

\newcommand{\pp}[0]{\mathbb{P}}
\newcommand{\dbm}[0]{{
		[1,\partial_{b},\partial_{m}]
}}

\newcommand{\dg}[0]{{[1, \partial_{\theta_G}]}}
\newcommand{\dd}[0]{{[1, \partial_{\theta_D}]}}
\newcommand{\dgd}[0]{{[1, \partial_{\theta_G}, \partial_{\theta_D}]}}

\newcommand{\veca}[0]{{\vec{a}}}
\newcommand{\vecb}[0]{{\vec{b}}}
\newcommand{\vecc}[0]{{\vec{c}}}
\newcommand{\vecd}[0]{{\vec{d}}}
\newcommand{\vecf}[0]{{\vec{f}}}
\newcommand{\vech}[0]{{\vec{h}}}
\newcommand{\vecr}[0]{{\vec{r}}}
\newcommand{\vecs}[0]{{\vec{s}}}
\newcommand{\vecu}[0]{{\vec{u}}}
\newcommand{\vecx}[0]{{\vec{x}}}
\newcommand{\vecy}[0]{{\vec{y}}}
\newcommand{\vechy}[0]{{\vec{\haty}}}
\newcommand{\vtheta}[0]{{\vec{\theta}}}

\newcommand{\haty}[0]{{\hat{y}}}
\newcommand{\hata}[0]{{\hat{a}}}
\newcommand{\hatr}[0]{{\hat{r}}}


\newcommand{\cond}[0]{{\:\mathbf{|}\:}}
\newcommand{\mymathbf}[1]{#1}

\newcommand{\ranvec}[1]{\ul{\vec{#1}}}
\newcommand{\indi}[0]{\mathbbm{1}}
\newcommand{\smoid}[0]{{\rm smoid}}
\newcommand{\lodds}[0]{{\rm lodds}}
\newcommand{\sign}[0]{{\rm sign}}

\renewcommand{\labelitemii}{$\bullet$}

\newcommand{\tilx}[0]{{\tilde{x}}}
\newcommand{\tile}[0]{{\tilde{e}}}
\newcommand{\tilu}[0]{{\tilde{u}}}

\newcommand{\ucalm}[0]{\ul{\calm}}

\newcommand{\bool}[0]{\{0,1\}}

\newcommand{\argmin}{\mathop{\mathrm{argmin}}\limits}
\newcommand{\argmax}{\mathop{\mathrm{argmax}}\limits}  

\newcommand{\softmax}[0]{{\rm softmax}}

\newcommand{\A}[0]{\wedge}
\newcommand{\V}[0]{\vee}

\newcommand{\rdart}[0]{\Rightarrow}
\newcommand{\ldart}[0]{\Leftarrow}
\newcommand{\rveps}[0]{\ul{\eps}}

\newcommand{\hatvar}[0]{\widehat{\sigma^2}}
\newcommand{\ptp}[0]{{(t)}}

%arguments phi1, phi2, phi3, e
\newcommand{\rbd}[4]{
\xymatrix@-1.3pc{
&#1\ar[d]&&#2\ar[d]
\\
&\rvx_1\ar[rr]
&&
\rvx_2
\ar `r[rd][rd]
\\
0\ar`u[u][ru]
\ar`d[dr][rrd]
&&&&\rvA\ar[r]&#4
\\
&&
\rvx_3
\ar `r[rru][rru]
\\
&&#3\ar[u]
}
}


\newcommand{\ruleone}[0]{
If $(\rvb. \perp \rva. 
|\rvr., \rvs.)$
in $\rho_{\rvr.} G$, then
$
P(b.|a., \rho\rvr.=r.,s.)=
P(b.|\rho\rvr.=r., s.)
$.
}

\newcommand{\ruletwo}[0]{
If $(\rvb.\perp \rva. |
 \rvr., \rvs.)$
in $\lam_{\rva.}\rho_{\rvr.} G$, 
then $
P(b.|\rho\rva.=a., \rho\rvr.=r., s.)=
P(b.|a., \rho\rvr.=r.,  s.)
$.
}

\newcommand{\rulethree}[0]{
If $
(\rvb. \perp
\rva.
| \rvr., \rvs.)$
in $
\rho_{\rva.-an(\rvs.)}
\rho_{\rvr.}G$,
 then
$
P(b.|\rho\rva.=a.,\rho\rvr.=r.,  s.)=
P(b.|\rho\rvr.=r., s.)$.
}

\newcommand{\bdoordef}[0]{
Suppose that we have access to data
that allows us to 
estimate a probability
distribution 
 $P(x., y., z.)$.
Hence, the variables
$\rvx., \rvy., \rvz.$ are 
ALL the observed (i.e, not hidden).
Then we say that the backdoor
$\rvz.$ satisfies the 
{\bf backdoor adjustment criterion}
relative to $(\rvx., \rvy.)$
if
\begin{enumerate}
\item 
All backdoor paths from $\rvx.$ 
to $\rvy.$ 
 are blocked by  
 $\rvz.$.
\item 
$\rvz. \cap de(\rvx.)=\emptyset$.
\end{enumerate}
}

\newcommand{\bdoorclaim}[0]{
If $\rvz.$ satisfies the 
backdoor criterion relative to
 $(\rvx., \rvy.)$, then

\beqa
P(y. | \rho \rvx.=x.)&=&
\sum_{z.} P(y.|x., z.)P(z.)
\\
&=&
\begin{array}{l}
\\
\\
\sum_{z.}
\end{array}
\xymatrix{
\rvz.=z.\ar[dr]
\\
\rvx.=x.\ar[r]&\rvy.
}
\eeqa
}

\newcommand{\fdoordef}[0]{
Suppose that we have access to data
that allows us to 
estimate a probability
distribution 
 $P(x., m., y.)$.
Hence, the variables
$\rvx., \rvm., \rvy.$ are 
ALL the observed (i.e, not hidden).
Then we say that the front-door
$\rvm.$ satisfies the 
{\bf front-door adjustment criterion} relative
 to $(\rvx., \rvy.)$
if
\begin{enumerate}
\item 
All directed paths from
$\rvx.$ to $\rvy.$ are intercepted by
(i.e., have a node in) $\rvm.$.
\item 
All backdoor paths from $\rvx.$ to
$\rvm.$ are blocked (by $\emptyset$).
\item 
All backdoor paths from
on $\rvm.$ to $\rvy.$
are blocked by $\rvx.$.
\end{enumerate}
}

\newcommand{\fdoorclaim}[0]{
If $\rvm.$ satisfies the 
front-door criterion 
relative to $(\rvx., \rvy.)$,
and $P(x.,m.)>0$, then

\beqa
P(y. | \rho \rvx.=x.)&=&\sum_{m.}
\underbrace{\left[\sum_{x.'}
P(y.|x'., m.)P(x'.)\right]}_
{P(y.|\rho \rvm.=m.)} 
\underbrace{P(m.|x.)}_
{P(m.|\rho \rvx.=x.)}
\\
&=&
\begin{array}{l}
\\
\\
\sum_{m., x.'}
\end{array}
\xymatrix{
&\rvx.=x.'\ar[dr]
\\
\rvx.=x.´\ar[r]
&\rvm.=m.\ar[r]&\rvy.
}
\eeqa
}


%Symmetry
\newcommand{\symrule}[0]{
$\rva\perp_P\rvb\implies \rvb\perp_P\rva$}

\newcommand{\symruleH}[0]{
$H(\rva:\rvb)=0\implies H(\rvb:\rva)=0$}

%Decomposition
\newcommand{\decrule}[0]{
$\rva\perp_P\rvb, \rvc\implies 
\rva\perp_P\rvb \text{ and } \rva\perp_P\rvc$}

\newcommand{\decruleH}[0]{
$H(\rva:\rvb, \rvc)=0\implies 
H(\rva:\rvb)=0 \text{ and } H(\rva:\rvc)=0$}

%Weak Union
\newcommand{\wearule}[0]{
$\rva\perp_P \rvb, \rvc \implies 
\rva\perp_P\rvb|\rvc\text{ and }\rva\perp_P\rvc|\rvb$}

\newcommand{\wearuleH}[0]{
$H(\rva:\rvb, \rvc)=0 \implies 
H(\rva:\rvb|\rvc)=0\text{ and }H(\rva:\rvc|\rvb)=0$}

%Contraction
\newcommand{\conrule}[0]{
$\rva\perp_P\rvb|\rvc\text{ and }\rva\perp_P \rvc
\implies \rva\perp_P \rvb, \rvc$}

\newcommand{\conruleH}[0]{
$H(\rva:\rvb|\rvc)=0\text{ and }H(\rva:\rvc)=0
\implies H(\rva:\rvb, \rvc)=0$}

%Intersection
\newcommand{\intrule}[0]{
$\rva\perp_P\rvb|\rvc, \rvd\text{ and }
\rva\perp_P \rvd|\rvc, \rvb\implies
\rva\perp_P \rvb,\rvd|\rvc$}

\newcommand{\intruleH}[0]{
$H(\rva:\rvb|\rvc, \rvd)=0\text{ and }
H(\rva:\rvd|\rvc, \rvb)=0\implies
H(\rva:\rvb,\rvd|\rvc)=0$}

\newcommand{\dotbarmu}[0]{{\cdot|\mu}}
\newcommand{\dotmu}[0]{{\cdot, \mu}}
\newcommand{\kbarmu}[0]{{k|\mu}}
\newcommand{\kmu}[0]{{k,\mu}}
\newcommand{\plusbarmu}[0]{{+|\mu}}
\newcommand{\plusmu}[0]{{+,\mu}}

\newcommand{\bnlearn}[0]{{\tt bnlearn\;}}

\newcommand{\sqsig}[0]{{[\sigma]}}

\newcommand{\misscellone}[0]{
\begin{array}{c}
\frac{1}{nsam}
P(x_0=0, x_2=0\cond x_1=1, \theta)
\\
\frac{1}{nsam}
P(x_0=0, x_2=1\cond x_1=1, \theta)
\\
\frac{1}{nsam}
P(x_0=1, x_2=0\cond x_1=1, \theta)
\\
\frac{1}{nsam}
P(x_0=1, x_2=1\cond x_1=1, \theta)
\end{array}
}

\newcommand{\misscelltwo}[0]{
\begin{array}{c}
\frac{1}{nsam}
P(x_1=0\cond x_0=0,x_2=1, \theta)
\\
\frac{1}{nsam}
P(x_1=1\cond x_0=0,x_2=1,  \theta)
\end{array}
}

\newcommand{\tD}[0]{{\tilde{D}}}
\newcommand{\td}[0]{{\tilde{d}}}
\newcommand{\rvtd}[0]{{\ul{\tilde{d}}}}
\newcommand{\tx}[0]{{\tilde{x}}}
\newcommand{\rvtx}[0]{{\ul{\tilde{x}}}}

\newcommand{\mlarr}[0]{\xrightarrow{\rm ML-fit}}
\newcommand{\lrarr}[0]{\xrightarrow{\rm LR-fit}}

\newcommand{\setprob}[3]
{{\begin{array}{c}S=\{#1\}
\\P(S)=#2\\ \haty(x^\s_S)=\$#3 K
\end{array}}}

\begin{document}
\title{Is Rubin's Potential 
Outcomes Theory Well Defined?}
\date{ \today}
\author{Robert R. Tucci\\
        tucci@ar-tiste.com}
\maketitle
\vskip2cm
\section*{Abstract}
Donald Rubin's Potential Outcomes
 theory makes 
two key assumptions that we
shall call SUTVA and CIA.
In this brief letter, we 
question whether those two
assumptions can hold 
simultaneously.

\newpage
\section{Introduction}
Donald Rubin's Potential Outcomes (PO) 
theory (a.k.a. Rubin's Causal Model) 
(Ref. \cite{rubin2005}) is a
popular method for 
doing causal inference (CI).
PO theory is explained in numerous 
textbooks (Refs.\cite{alves-book, book-mixtape, bayesuvius}).

PO theory makes 
two key assumptions that we
shall call SUTVA and CIA.
In this brief letter, we 
question whether those two
assumptions can hold 
simultaneously.


\section{Standard PO Assumptions}
Standard PO analysis 
considers
 random variables
$D^\s\in\bool$, $X^\s$, $Y^\s$ and
$\vec{\; Y^\s}=(Y^\s(0), Y^\s(1))$,
where index $\s$
labels the
members (individuals, units)
of the
population (dataset)
being considered.
These variables are
constrained by the following 2 assumptions:
\begin{enumerate}
\item SUTVA
\beq
Y^\s=
D^\s Y^\s(1) + (1-D^\s)Y^\s(0)
\eeq
\item 
Conditional Independence Assumption (CIA)

\beq
Y^\s(0), Y^\s(1)\perp D^\s|X^\s
\eeq
\end{enumerate}

By virtue of these 2 assumptions, 
we have, for $d\in \bool$,

\begin{subequations}
\label{eq-expect-sutva-cia}
\beqa
E[Y^\s|D^\s=d, X^\s]
&=&
E[Y^\s(d)|D^\s=d, X^\s]
\;\;\;\text{(by SUTVA)}
\label{eq-SUTVA}
\\
&=&
E[Y^\s(d)|X^\s]
\;\;\;
\text{(by CIA)}
\label{eq-cia}
\eeqa
\end{subequations}

In standard PO theory, one defines the
Average Treatment Effect (ATE) by

\beq
ATE \eqdef E[Y^\s(1)-Y^\s(0)]
\eeq
and its $x$ stratum by

\beq
ATE_x\eqdef E[Y^\s(1)-Y^\s(0)|X^\s=x]
\;
\label{eq-ate-x-def}
\eeq
so that

\beq
ATE =
\sum_x P(x) ATE_x
\;.
\eeq
$ACE_x$ is defined by 
Eq.(\ref{eq-ate-x-def}),
but by virtue of
 Eq.\ref{eq-expect-sutva-cia},
it also equals
\beq
ATE_x=
E[Y^\s|D^\s=1, X^\s]-
E[Y^\s|D^\s=0, X^\s]
\label{eq-ace-x-2}
\eeq


\section{Can CIA and SUTVA be satisfied
simultaneously?}

Throughout the previous section,
and in
particular in 
Eqs.(\ref{eq-expect-sutva-cia})
and (\ref{eq-ace-x-2}),
we assumed that CIA and SUTVA
can hold simultaneously.
Assuming this is
standard 
practice in PO 
theory.
In this section,
we question 
whether that
assumption
can ever hold.





\begin{figure}[h!]
$$
\begin{array}{ccccc}
\xymatrix{
&X^\s\ar[ddl]\ar[ddr]
\\
\\
D^\s\ar[rr]&&Y^\s
}
&
\xymatrix{
&X^\s\ar[ddl]\ar[dr]
\\
&&[Y^\s(0),Y^\s(1)]\ar[d]
\\
D^\s
\ar[rr]
\ar[urr]&&Y^\s
}
&
\xymatrix{
&X^\s\ar[ddl]\ar[dr]
\\
&&[Y^\s(0),Y^\s(1)]\ar[d]
\\
D^\s&\tD^\s\ar[r]\ar[ur]
&Y^\s
}
\\
\\
G_0&G_1&G_2
\end{array}
$$
\caption{Three Bayesian networks (bnets)
that could possibly describe PO theory.
} 
\label{fig-po-G-im-y0-y1}
\end{figure}

Fig.\ref{fig-po-G-im-y0-y1}
shows 3 Bayesian networks\footnote{Bayesian 
networks are extensively
discussed 
by the author of this paper
in his textbook Ref.\cite{bayesuvius}
} (bnets)
labeled   $G_0, G_1, G_2$
that could possibly
describe PO theory.\footnote{Remember
 that bnets are 
merely a graphical
representation
of the chain rule
for conditional
probabilities.
Our using bnets
in this paper
does not 
constitute 
assuming anything
beyond the axioms
of standard probability theory.}
The Transition
Probability Matrices (TPMs),
printed in blue,
for the nodes of these 3 bnets, are
as follows:

\begin{itemize} 

\item TPMs for $G_0$

\begin{subequations}
\beq \color{blue}
P(x^\s)= P_X(x^\s)
\eeq

\beq \color{blue}
P(d^\s|x^\s)= P_{D|X}(d^\s|x^\s)
\eeq

\beq \color{blue}
P(y^\s|d^\s,x^\s)= P_{Y|D,X}(y^\s|d^\s, x^\s)
\eeq
\end{subequations}

\item TPMs for $G_1$

\begin{subequations}
\beq \color{blue}
P(x^\s)= P_X(x^\s)
\eeq

\beq \color{blue}
P(d^\s|x^\s)= P_{D|X}(d^\s|x^\s)
\eeq


\beq \color{blue}
P(y^\s|d^\s,
\vec{\;y}^\s)= 
\indi(
y^\s= y^\s(d^\s))
\label{eq-indi-g1}
\eeq


For $c\in \bool$, 
\beq \color{blue}
P(y^\s(c)|d^\s,x^\s)=
P_{Y(c)|D, X}(y^\s(c)|d^\s, x^\s)
\eeq

\end{subequations}

\item TPMs for $G_2$

\begin{subequations}
\beq \color{blue}
P(x^\s)= P_X(x^\s)
\eeq

\beq \color{blue}
P(d^\s|x^\s)= P_{D|X}(d^\s|x^\s)
\eeq


\beq \color{blue}
P(y^\s|\td^\s,
\vec{\;y}^\s)= 
\indi(
y^\s= y^\s(\td^\s))
\label{eq-indi-g2}
\eeq

For $c\in \bool$,
\beq \color{blue}
P(y^\s(c)|\td^\s,x^\s)=
P_{Y(c)|\tD, X}(y^\s(c)|\td^\s, x^\s)
\eeq
\end{subequations}

\end{itemize}




{\renewcommand{\arraystretch}{1.5}
\begin{table}[h!]
\centering
\begin{tabular}{|l|l|l|l|}
\hline
\rowcolor[HTML]{ECF4FF} 
 PO assumption & $G_0?$
& $G_1?$ & $G_2?$
\\ \hline
\mbox{\footnotesize $E[Y^\s(1)|D^\s=1, X^\s]
=
E[Y^\s(1)|X^\s]$ (CIA)}
&NA  &No  &  Yes
\\ \hline
\mbox{\footnotesize$E[Y^\s|D^\s=1, X^\s]
=
E[Y^\s(1)|D^\s=1, X^\s]$ (SUTVA)}
&NA  & Yes 
\mbox{\tiny (Eq.(\ref{eq-indi-g1}))}
 & No
\\ \hline
\mbox{\footnotesize$E[Y^\s|\tD^\s=1, X^\s]
=
E[Y^\s(1)|\tD^\s=1, X^\s]$ (SUTVA$\sim$)}
 & NA &NA  & 
Yes \mbox{\tiny (Eq.(\ref{eq-indi-g2}))}  
\\ \hline
\end{tabular}
\caption{
``NA" means not applicable.
``Yes" means that the graph
satisfies the PO assumption,
and ``No" means that it doesn't.
}
\label{tab-3graphs}
\end{table}

Now consider Table \ref{tab-3graphs}.
In that table,
\begin{itemize}
\item
$G_0?$ is NA for all 3 PO assumptions
because $G_0$
does not contain nodes for $Y^\s(0)$
and $Y^\s(1)$
and these
appear in the
3 PO assumptions.
\item
The $D$ in SUTVA
is replaced by a $\tD$ in  SUTVA$\sim$.

\item
$G_1?$
is NA 
for 
SUTVA$\sim$
because
$G_1$
doesn't have a $\tD^\s$ node.
\item
The entries for the CIA row 
are a consequence of
Pearl's d-separation theorem.

\item Two ``Yes" entries
are justified by 
referring to an equation.
\end{itemize}


As told by Table \ref{tab-3graphs},
$G_0, G_1$ and $G_2$
all violate either
SUTVA or CIA.
$G_2$ 
doesn't satisfy both CIA and SUTVA, 
but it does
satisfy CIA and
a modified version
of SUTVA that we call
SUTVA$\sim$.

\bibliographystyle{plain}
\bibliography{references}
\end{document}
